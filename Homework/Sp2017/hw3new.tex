\documentclass[12pt]{article}
%%%%%%%%%%%%%%%%%%%%%%%%%%%%%%%%%%%%%%%%%%%%%%%%%%%%%%%%%%%%%
%% Document setup for Syllabus and HW

\textwidth=7in
\textheight=9.5in
\topmargin=-1in
\headheight=0in
\headsep=.5in
\hoffset  -.85in

\usepackage{amsmath} % for \text{}
\usepackage{amsfonts} % For \text{}
\usepackage{amssymb} % For \text{}
\usepackage{nth}
\usepackage{enumerate}
%\usepackage{nth}

%% Stolen from Nathan Baker
\newcommand{\erf}{{\mathrm{erf}}}
\newcommand{\erfc}{{\mathrm{erfc}}}
\newcommand{\erfi}{{\mathrm{erfi}}}
\newcommand{\argh}{{\mathrm{arg}}}
\newcommand{\atan}{{\mathrm{atan}}}
\newcommand{\acos}{{\mathrm{acos}}}
\newcommand{\asin}{{\mathrm{asin}}}
\newcommand{\mat}[1]{\,\underline{\underline{#1}}\,}
\newcommand{\abs}[1]{\left| #1 \right|}
\newcommand{\norm}[1]{\left\| #1 \right\|}
\newcommand{\order}[1]{{\mathcal{O}} \left( #1 \right)}
\newcommand{\op}[1]{{\mathcal{#1}}}
\newcommand{\myfile}[1]{\texttt{#1}}
\newcommand{\myvar}[1]{\textsf{#1}}
\newcommand{\mean}[1]{{\left\langle {#1} \right\rangle}}

% The master boolean -- set this to "masterfalse" or "mastertrue"
%\newif\ifmaster \masterfalse
\newif\ifmaster \masterfalse
\newif\ifmaster \mastertrue

%http://tex.stackexchange.com/questions/145812/using-fbox-in-a-newenvironment
\newcommand\notes[1]{%
  \fbox{\begin{minipage}{0.9\textwidth}#1\end{minipage}}}

%%\newcommand{\hidesolution}[1]{ \ifmaster { {#1} } \else { {Go to class.} } \fi }
%%\newcommand{\hideanswer}[1]{ \ifmaster { #1 } \else {} \fi }
%%\newcommand{\solution}[1]{ \begin{proof}[Solution] \hidesolution{#1} \end{proof} }
%%\newcommand{\answer}[1]{ \hideanswer{\newpage\ \newpage \notes{\begin{proof}[Answer] #1 \end{proof}}} }

%% MGL
\newcommand{\CC}{\mathbb{C}}
\newcommand{\N}{\mathbb{N}}
%\newcommand{\vect}[1]{\,\vec{\mathbf{#1}}\,}
\newcommand{\vect}[1]{\,\mathbf{#1}\,}
\usepackage{tabularx}
\newcommand{\vtau}{\vect{\tau}}
\newcommand{\va}{\vect{a}}
\newcommand{\vb}{\vect{b}}
\newcommand{\vc}{\vect{c}}
\newcommand{\vd}{\vect{d}}
\newcommand{\ve}{\vect{e}}
\newcommand{\vf}{\vect{f}}
\newcommand{\vg}{\vect{g}}
\newcommand{\vh}{\vect{h}}
\newcommand{\vi}{\vect{i}}
\newcommand{\vj}{\vect{j}}
\newcommand{\vk}{\vect{k}}
\newcommand{\vl}{\vect{l}}
\newcommand{\vm}{\vect{m}}
\newcommand{\vn}{\vect{n}}
\newcommand{\vo}{\vect{o}}
\newcommand{\vp}{\vect{p}}
\newcommand{\vq}{\vect{q}}
\newcommand{\vr}{\vect{r}}
\newcommand{\vs}{\vect{s}}
\newcommand{\vt}{\vect{t}}
\newcommand{\vu}{\vect{u}}
\newcommand{\vv}{\vect{v}}
\newcommand{\vw}{\vect{w}}
\newcommand{\vx}{\vect{x}}
\newcommand{\vy}{\vect{y}}
\newcommand{\vz}{\vect{z}}

\newcommand{\vA}{\vect{A}}
\newcommand{\vB}{\vect{B}}
\newcommand{\vC}{\vect{C}}
\newcommand{\vD}{\vect{D}}
\newcommand{\vE}{\vect{E}}
\newcommand{\vF}{\vect{F}}
\newcommand{\vG}{\vect{G}}
\newcommand{\vH}{\vect{H}}
\newcommand{\vI}{\vect{I}}
\newcommand{\vJ}{\vect{J}}
\newcommand{\vK}{\vect{K}}
\newcommand{\vL}{\vect{L}}
\newcommand{\vM}{\vect{M}}
\newcommand{\vN}{\vect{N}}
\newcommand{\vO}{\vect{O}}
\newcommand{\vP}{\vect{P}}
\newcommand{\vQ}{\vect{Q}}
\newcommand{\vR}{\vect{R}}
\newcommand{\vS}{\vect{S}}
\newcommand{\vT}{\vect{T}}
\newcommand{\vU}{\vect{U}}
\newcommand{\vV}{\vect{V}}
\newcommand{\vW}{\vect{W}}
\newcommand{\vX}{\vect{X}}
\newcommand{\vY}{\vect{Y}}
\newcommand{\vZ}{\vect{Z}}
\newcommand{\diffd}{\mathrm{d}}
\newcommand{\deriv}[2]{\frac{\diffd #1}{\diffd #2}} % derivative
%\newcommand{\pd}[2]{\frac{\partial#1}{\partial#2}}
%\newcommand{\dd}[2]{\frac{d#1}{d#2}}
\newcommand{\dinline}[2]{\diffd #1/\diffd #2} % for derivatives
\newcommand{\pdinline}[2]{\partial#1/\partial#2}

%\newcommand{\bra}[1]{\left\langle #1 \right|}
%\newcommand{\ket}[1]{\left| #1 \right\rangle}

\newcommand{\xd}{\dot{x}}

\newcommand{\boldcent}[1] {\begin{center}\textbf{ #1 }\end{center}}
% ***********************************************************
% ******************* PHYSICS HEADER ************************
% ***********************************************************
% From http://www.dfcd.net/articles/latex/latex.html
% Version 2
%\documentclass[11pt]{article} 
\usepackage{amsmath} % AMS Math Package
\usepackage{amsthm} % Theorem Formatting
\usepackage{amssymb}	% Math symbols such as \mathbb
\usepackage{graphicx} % Allows for eps images
\usepackage{multicol} % Allows for multiple columns
%\usepackage[dvips,letterpaper,margin=0.75in,bottom=0.5in]{geometry}
% % Sets margins and page size
%\pagestyle{empty} % Removes page numbers
%\makeatletter % Need for anything that contains an @ command 
%\renewcommand{\maketitle} % Redefine maketitle to conserve space
%{ \begingroup \vskip 10pt \begin{center} \large {\bf \@title}
%	\vskip 10pt \large \@author \hskip 20pt \@date \end{center}
%  \vskip 10pt \endgroup \setcounter{footnote}{0} }
%\makeatother % End of region containing @ commands
\renewcommand{\labelenumi}{(\alph{enumi})} % Use letters for enumerate
% \DeclareMathOperator{\Sample}{Sample}
\let\vaccent=\v % rename builtin command \v{} to \vaccent{}
\renewcommand{\v}[1]{\ensuremath{\mathbf{#1}}} % for vectors
\newcommand{\gv}[1]{\ensuremath{\mbox{\boldmath$ #1 $}}} 
% for vectors of Greek letters
\newcommand{\uv}[1]{\ensuremath{\mathbf{\hat{#1}}}} % for unit vector
%\newcommand{\abs}[1]{\left| #1 \right|} % for absolute value
\newcommand{\avg}[1]{\left< #1 \right>} % for average
\let\underdot=\d % rename builtin command \d{} to \underdot{}
\newcommand{\fd}[2]{\frac{d #1}{d #2}} % for derivatives
\newcommand{\dd}[2]{\frac{d^2 #1}{d #2^2}} % for double derivatives
\newcommand{\pd}[2]{\frac{\partial #1}{\partial #2}} 
% for partial derivatives
\newcommand{\pdd}[2]{\frac{\partial^2 #1}{\partial #2^2}} 
% for double partial derivatives
\newcommand{\pdc}[3]{\left( \frac{\partial #1}{\partial #2}
 \right)_{#3}} % for thermodynamic partial derivatives
\newcommand{\ket}[1]{\left| #1 \right>} % for Dirac bras
\newcommand{\bra}[1]{\left< #1 \right|} % for Dirac kets
\newcommand{\braket}[2]{\left< #1 \vphantom{#2} \right|
 \left. #2 \vphantom{#1} \right>} % for Dirac brackets
\newcommand{\matrixel}[3]{\left< #1 \vphantom{#2#3} \right|
 #2 \left| #3 \vphantom{#1#2} \right>} % for Dirac matrix elements
\newcommand{\grad}[1]{\gv{\nabla} #1} % for gradient
\let\divsymb=\div % rename builtin command \div to \divsymb
\renewcommand{\div}[1]{\gv{\nabla} \cdot #1} % for divergence
\newcommand{\curl}[1]{\gv{\nabla} \times #1} % for curl
\let\baraccent=\= % rename builtin command \= to \baraccent
\renewcommand{\=}[1]{\stackrel{#1}{=}} % for putting numbers above =
\newtheorem{prop}{Proposition}
\newtheorem{thm}{Theorem}[section]
\newtheorem{lem}[thm]{Lemma}
\theoremstyle{definition}
\newtheorem{dfn}{Definition}
\theoremstyle{remark}
\newtheorem*{rmk}{Remark}

% ***********************************************************
% ********************** END HEADER *************************
% ***********************************************************

%%%%%%%%%%%%%%%%%%%%%%%%%%%%%%%%%%%%%%%%%%%%%%%%%%%%%%%%%%%%%

\pagestyle{empty}
\begin{document}
\begin{center}
{\bf Physics 360/Math 360  \ \ MF 12:00 - 12:50 PM,  W 2:30-3:20, Room:  CST 314
}
\end{center}

\setlength{\unitlength}{1in}

\begin{picture}(6,.1) 
\put(0,0) {\line(1,0){6.25}}         
\end{picture}


\vskip.15in
\noindent\textbf{Instructor:} Michael Lerner,  CST 213 221, Phone: 727-LERNERM
\vskip.15in
\makebox[\textwidth]{Name:\enspace\hrulefill}
\vskip.15in

\noindent\textbf{Assignment 3, Due as specified}
\vskip.15in

\noindent {\bf For Friday Jan 27}

Using a table like the one in Boas \S 12.2, solve problems 12.1.1
and 12.1.4

\vskip.3in
\noindent {\bf For Wednesday Feb 1}

In this assignment, as in many of the future assignments, we will
investigate a particular topic in depth, rather than solving several
separate problems. This week, we focus on the Legendre polynomials,
which have broad applicability in mathematical physics, especially in
the modeling of spherically symmetric systems.

The text of the following problems is taken (with some small changes)
from Boyce and DiPrima, Chapter 5, section 3.

The following problems deal with the Legendre equation:


\begin{equation}
  \label{L}
  (1-x^2)y'' - 2xy' + \alpha(\alpha+1)y = 0
\end{equation}

Following the convention of choosing a fundamental set of solutions such that

\begin{align*}
  y_1(x) &= 1 + b_2(x-x_0)^2 + ... \\
  y_2(x) &= (x-x_0) + c_2 (x-x_0)^3 + ... \\
  b_2+c_2 &= a_2
\end{align*}

(Note that these series have already included the fact that one will
be even and one will be odd, a fact that you'll show below.)

Two solutions of the Legendre equation for $|x| < 1$ are

\begin{align*}
  y_1(x) &= 1 - \frac{\alpha(\alpha+1)}{2!}x^2 + \frac{\alpha(\alpha-2)(\alpha+1)(\alpha+3)}{4!}x^4 \\
   &+ \sum_{m=3}^{\infty}(-1)^m\frac{\alpha\cdot\cdot\cdot(\alpha-2m+2)(\alpha+1)\cdot\cdot\cdot(\alpha+2m-1)}{(2m)!}x^{2m}, \\
  y_2(x) &= x - \frac{(\alpha - 1)(\alpha + 2)}{3!}x^3 + \frac{(\alpha-1)(\alpha-3)(\alpha+2)(\alpha+4)}{5!}x^5 \\
  &+ \sum_{m=3}^{\infty}\frac{(\alpha-1)\cdot\cdot\cdot(\alpha-2m+1)(\alpha+2)\cdot\cdot\cdot(\alpha+2m)}{(2m+1)!}x^{2m+1}
\end{align*}

\vskip.1in
\textbf{Problem 1}
Write out the first 4 terms for $y_1$ and $y_2$.

\vskip.1in
\textbf{Problem 2} Show that, if $\alpha$ is zero or a positive even
integer $2n$, the series solution $y_1$ reduces to a polynomial of
degree $2n$ containing only even powers of $x$. Find the polynomials
corresponding to $\alpha=0,2,4$. Similarly, show that if $\alpha$ is a
positive odd integer $2n+1$, the series solution $y_2$ reduces to a
polynomial of degree $2n+1$ containing only odd powers of $x$. Find
the polynomials corresponding to $\alpha=1,3,5$. 

\vskip.1in \textbf{Problem 3} The Legendre polynomial $P_n(x)$ is
defined as the polynomial solution of the Legendre equation with
$\alpha=n$ that also satisfies the condition
$P_n(1)=1$.\\  

\textbf{(a)} Using the results of Problem 2, find the
first five Legendre polynomials, $P_0(x),...,P_5(x)$. \\  

\textbf{(b)}
Plot the graphs of $P_0(x),...,P_5(x)$ in the range for which we've
demonstrated convergence, $|x|\le1$. You may use whatever graphing
package you'd like, including but not limited to Wolfram alpha
(e.g. go to wolframalpha.com and type 
``plot 0.5*(3x\verb|^|2-1) from -1 to
1'' in the box), Matlab, Mathematica, Python+matplotlib, etc. \\

\textbf{(c)} Find the zeros of $P_0(x),...,P_5(x)$.


\vskip.1in
\textbf{Problem 4} The Legendre polynomials play an important role in
mathematical physics. For example, solving the potential equation
(Laplace's equation) in spherical coordinates, we encounter the
equation 


\begin{equation*}
  \dd{F(\phi)}{\phi} + \cot\phi\fd{F(\phi)}{\phi} + n(n+1)F(\phi) = 0,\qquad 0 < \phi < \pi
\end{equation*}

Show that the change of variables $x=\cos\phi$ leads to the Legendre
equation with $\alpha=n$ for $y=f(x)=F(\cos^{-1}(x))$ 

Hint: you may need to use the fact that


\begin{equation*}
  \sin(\arccos(x)) = \sqrt{1-x^2};\qquad \cot(\arccos(x)) = \frac{x}{\sqrt{1-x^2}}
\end{equation*}

\vskip.3in
\noindent {\bf For Friday Feb 3}
\vskip.1in
\textbf{Problem 5} Show that the Legendre equation can also be written as

\begin{equation*}
  [(1-x^2)y']' = -\alpha(\alpha+1)y
\end{equation*}

It then follows that

\begin{equation}
  \label{e1}
  [(1-x^2)P_n^{'}(x)]' = -n(n+1)P_n(x)
\end{equation}
and
\begin{equation}
  \label{e2}
  [(1-x^2)P_m^{'}(x)]' = -m(m+1)P_m(x).
\end{equation}
By multiplying \eqref{e1} by $P_m(x)$ and \eqref{e2} by $P_n(x)$,
\textbf{integrating by parts}, and then subtracting one equation from the
other, show that


\begin{equation}
  \label{orth}
  \int_{-1}^{1}P_n(x)P_m(x)dx = 0 \qquad \mathrm{if} \quad n \ne m
\end{equation}

This property \eqref{orth} of the Legendre polynomials is known as the
orthogonality property. If $m=n$, it can be shown that the value of
the integral in \eqref{orth} is $2/(2n+1)$. 

\vskip.1in
Given a polynomial $f$ of degree $n$, it is possible to express $f$ as a linear combination of $P_0,P_1,...,P_n$:

\begin{equation}
  \label{basis}
  f(x) = \sum_{k=0}^n a_kP_k(x)
\end{equation}

Note that, since the $n+1$ polynomials $P_0,...,P_n$ are linearly
independent, and the degree of $P_k$ is $k$, any polynomial of degree
$n$ can be expressed as \eqref{basis}. \\ Using the result of Problem
7, you can show that 


\begin{equation*}
  a_k=\frac{2k+1}{2}\int_{-1}^1f(x)P_k(x)dx
\end{equation*}

\textbf{but you don't have to! You're done!}

\end{document}
