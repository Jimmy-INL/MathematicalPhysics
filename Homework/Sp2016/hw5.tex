\documentclass[12pt]{article}
%%%%%%%%%%%%%%%%%%%%%%%%%%%%%%%%%%%%%%%%%%%%%%%%%%%%%%%%%%%%%
%% Document setup for Syllabus and HW

\textwidth=7in
\textheight=9.5in
\topmargin=-1in
\headheight=0in
\headsep=.5in
\hoffset  -.85in

\usepackage{amsmath} % for \text{}
\usepackage{amsfonts} % For \text{}
\usepackage{amssymb} % For \text{}
\usepackage{nth}
\usepackage{enumerate}
%\usepackage{nth}

%% Stolen from Nathan Baker
\newcommand{\erf}{{\mathrm{erf}}}
\newcommand{\erfc}{{\mathrm{erfc}}}
\newcommand{\erfi}{{\mathrm{erfi}}}
\newcommand{\argh}{{\mathrm{arg}}}
\newcommand{\atan}{{\mathrm{atan}}}
\newcommand{\acos}{{\mathrm{acos}}}
\newcommand{\asin}{{\mathrm{asin}}}
\newcommand{\mat}[1]{\,\underline{\underline{#1}}\,}
\newcommand{\abs}[1]{\left| #1 \right|}
\newcommand{\norm}[1]{\left\| #1 \right\|}
\newcommand{\order}[1]{{\mathcal{O}} \left( #1 \right)}
\newcommand{\op}[1]{{\mathcal{#1}}}
\newcommand{\myfile}[1]{\texttt{#1}}
\newcommand{\myvar}[1]{\textsf{#1}}
\newcommand{\mean}[1]{{\left\langle {#1} \right\rangle}}

% The master boolean -- set this to "masterfalse" or "mastertrue"
%\newif\ifmaster \masterfalse
\newif\ifmaster \masterfalse
\newif\ifmaster \mastertrue

%http://tex.stackexchange.com/questions/145812/using-fbox-in-a-newenvironment
\newcommand\notes[1]{%
  \fbox{\begin{minipage}{0.9\textwidth}#1\end{minipage}}}

%%\newcommand{\hidesolution}[1]{ \ifmaster { {#1} } \else { {Go to class.} } \fi }
%%\newcommand{\hideanswer}[1]{ \ifmaster { #1 } \else {} \fi }
%%\newcommand{\solution}[1]{ \begin{proof}[Solution] \hidesolution{#1} \end{proof} }
%%\newcommand{\answer}[1]{ \hideanswer{\newpage\ \newpage \notes{\begin{proof}[Answer] #1 \end{proof}}} }

%% MGL
\newcommand{\CC}{\mathbb{C}}
\newcommand{\N}{\mathbb{N}}
%\newcommand{\vect}[1]{\,\vec{\mathbf{#1}}\,}
\newcommand{\vect}[1]{\,\mathbf{#1}\,}
\usepackage{tabularx}
\newcommand{\vtau}{\vect{\tau}}
\newcommand{\va}{\vect{a}}
\newcommand{\vb}{\vect{b}}
\newcommand{\vc}{\vect{c}}
\newcommand{\vd}{\vect{d}}
\newcommand{\ve}{\vect{e}}
\newcommand{\vf}{\vect{f}}
\newcommand{\vg}{\vect{g}}
\newcommand{\vh}{\vect{h}}
\newcommand{\vi}{\vect{i}}
\newcommand{\vj}{\vect{j}}
\newcommand{\vk}{\vect{k}}
\newcommand{\vl}{\vect{l}}
\newcommand{\vm}{\vect{m}}
\newcommand{\vn}{\vect{n}}
\newcommand{\vo}{\vect{o}}
\newcommand{\vp}{\vect{p}}
\newcommand{\vq}{\vect{q}}
\newcommand{\vr}{\vect{r}}
\newcommand{\vs}{\vect{s}}
\newcommand{\vt}{\vect{t}}
\newcommand{\vu}{\vect{u}}
\newcommand{\vv}{\vect{v}}
\newcommand{\vw}{\vect{w}}
\newcommand{\vx}{\vect{x}}
\newcommand{\vy}{\vect{y}}
\newcommand{\vz}{\vect{z}}

\newcommand{\vA}{\vect{A}}
\newcommand{\vB}{\vect{B}}
\newcommand{\vC}{\vect{C}}
\newcommand{\vD}{\vect{D}}
\newcommand{\vE}{\vect{E}}
\newcommand{\vF}{\vect{F}}
\newcommand{\vG}{\vect{G}}
\newcommand{\vH}{\vect{H}}
\newcommand{\vI}{\vect{I}}
\newcommand{\vJ}{\vect{J}}
\newcommand{\vK}{\vect{K}}
\newcommand{\vL}{\vect{L}}
\newcommand{\vM}{\vect{M}}
\newcommand{\vN}{\vect{N}}
\newcommand{\vO}{\vect{O}}
\newcommand{\vP}{\vect{P}}
\newcommand{\vQ}{\vect{Q}}
\newcommand{\vR}{\vect{R}}
\newcommand{\vS}{\vect{S}}
\newcommand{\vT}{\vect{T}}
\newcommand{\vU}{\vect{U}}
\newcommand{\vV}{\vect{V}}
\newcommand{\vW}{\vect{W}}
\newcommand{\vX}{\vect{X}}
\newcommand{\vY}{\vect{Y}}
\newcommand{\vZ}{\vect{Z}}
\newcommand{\diffd}{\mathrm{d}}
\newcommand{\deriv}[2]{\frac{\diffd #1}{\diffd #2}} % derivative
%\newcommand{\pd}[2]{\frac{\partial#1}{\partial#2}}
%\newcommand{\dd}[2]{\frac{d#1}{d#2}}
\newcommand{\dinline}[2]{\diffd #1/\diffd #2} % for derivatives
\newcommand{\pdinline}[2]{\partial#1/\partial#2}

%\newcommand{\bra}[1]{\left\langle #1 \right|}
%\newcommand{\ket}[1]{\left| #1 \right\rangle}

\newcommand{\xd}{\dot{x}}

\newcommand{\boldcent}[1] {\begin{center}\textbf{ #1 }\end{center}}
% ***********************************************************
% ******************* PHYSICS HEADER ************************
% ***********************************************************
% From http://www.dfcd.net/articles/latex/latex.html
% Version 2
%\documentclass[11pt]{article} 
\usepackage{amsmath} % AMS Math Package
\usepackage{amsthm} % Theorem Formatting
\usepackage{amssymb}	% Math symbols such as \mathbb
\usepackage{graphicx} % Allows for eps images
\usepackage{multicol} % Allows for multiple columns
%\usepackage[dvips,letterpaper,margin=0.75in,bottom=0.5in]{geometry}
% % Sets margins and page size
%\pagestyle{empty} % Removes page numbers
%\makeatletter % Need for anything that contains an @ command 
%\renewcommand{\maketitle} % Redefine maketitle to conserve space
%{ \begingroup \vskip 10pt \begin{center} \large {\bf \@title}
%	\vskip 10pt \large \@author \hskip 20pt \@date \end{center}
%  \vskip 10pt \endgroup \setcounter{footnote}{0} }
%\makeatother % End of region containing @ commands
\renewcommand{\labelenumi}{(\alph{enumi})} % Use letters for enumerate
% \DeclareMathOperator{\Sample}{Sample}
\let\vaccent=\v % rename builtin command \v{} to \vaccent{}
\renewcommand{\v}[1]{\ensuremath{\mathbf{#1}}} % for vectors
\newcommand{\gv}[1]{\ensuremath{\mbox{\boldmath$ #1 $}}} 
% for vectors of Greek letters
\newcommand{\uv}[1]{\ensuremath{\mathbf{\hat{#1}}}} % for unit vector
%\newcommand{\abs}[1]{\left| #1 \right|} % for absolute value
\newcommand{\avg}[1]{\left< #1 \right>} % for average
\let\underdot=\d % rename builtin command \d{} to \underdot{}
\newcommand{\fd}[2]{\frac{d #1}{d #2}} % for derivatives
\newcommand{\dd}[2]{\frac{d^2 #1}{d #2^2}} % for double derivatives
\newcommand{\pd}[2]{\frac{\partial #1}{\partial #2}} 
% for partial derivatives
\newcommand{\pdd}[2]{\frac{\partial^2 #1}{\partial #2^2}} 
% for double partial derivatives
\newcommand{\pdc}[3]{\left( \frac{\partial #1}{\partial #2}
 \right)_{#3}} % for thermodynamic partial derivatives
\newcommand{\ket}[1]{\left| #1 \right>} % for Dirac bras
\newcommand{\bra}[1]{\left< #1 \right|} % for Dirac kets
\newcommand{\braket}[2]{\left< #1 \vphantom{#2} \right|
 \left. #2 \vphantom{#1} \right>} % for Dirac brackets
\newcommand{\matrixel}[3]{\left< #1 \vphantom{#2#3} \right|
 #2 \left| #3 \vphantom{#1#2} \right>} % for Dirac matrix elements
\newcommand{\grad}[1]{\gv{\nabla} #1} % for gradient
\let\divsymb=\div % rename builtin command \div to \divsymb
\renewcommand{\div}[1]{\gv{\nabla} \cdot #1} % for divergence
\newcommand{\curl}[1]{\gv{\nabla} \times #1} % for curl
\let\baraccent=\= % rename builtin command \= to \baraccent
\renewcommand{\=}[1]{\stackrel{#1}{=}} % for putting numbers above =
\newtheorem{prop}{Proposition}
\newtheorem{thm}{Theorem}[section]
\newtheorem{lem}[thm]{Lemma}
\theoremstyle{definition}
\newtheorem{dfn}{Definition}
\theoremstyle{remark}
\newtheorem*{rmk}{Remark}

% ***********************************************************
% ********************** END HEADER *************************
% ***********************************************************

%%%%%%%%%%%%%%%%%%%%%%%%%%%%%%%%%%%%%%%%%%%%%%%%%%%%%%%%%%%%%

%\pagestyle{empty}
\begin{document}
\title{The method of Frobenius summary; Bessel intro}
\author{Michael G. Lerner}
\date{\today}
\maketitle
%\begin{abstract}
%  A quick summary of the method of Frobenius and an introduction to Bessel functions
%\end{abstract}

\setlength{\unitlength}{1in}

\begin{picture}(6,.1) 
\put(0,0) {\line(1,0){6.25}}         
\end{picture}

\section{A couple of definitions}

Given a second-order differential equation 

\begin{equation}
P(x)y'' + Q(x)y' + R(x)y = 0,
\end{equation}

we'd like to be able to write
down a general solution. Sadly, the world is not kind enough to let us do
so. The method of Frobenius lets us write down solutions for most of
the points that we find interesting. First off, we can write it as

\begin{equation}
y'' + p(x)y' + p(x)y = 0
\end{equation}

where

\begin{equation}
p(x) \equiv Q(x)/P(x) \qquad q(x) \equiv R(X)\P(x).
\end{equation}

When doesn't this work? Well, clearly it doesn't work when $P(x)$ is
$0$. We call those {\bf singular points}. $p(x)$ or $q(x)$ diverge
there.\footnote{You have to be careful here. We're only in trouble if
  $p(x)$ or $q(x)$ diverge. So, if someone hands you $xy'' + xy' +
  x^2y = 9$, you don't call $0$ a singular point: although $P(x) = 0$,
  $p(x) = 1$ and $q(x) = x$.} 
We call points where
$p(x)$  and $q(x)$ do not diverge (i.e. $P(x) \ne 0$) {\bf ordinary points}. Near ordinary points, we can find
solutions with the power series methods we've used so far for, e.g.,
Airy's equation. However, when modeling physical systems, we find that
we often care about singular points. They show up at physical
boundaries, for instance. So, the reasoning goes, what if things
didn't blow up ``too quickly''? Could we write down a solution then? 

Consider a singular point $x_0$. This means that $p(x)$ diverges as $x
\to x_0$. If, however, $(x-x_0)p(x)$ and $(x-x_0)^2q(x)$ remain finite
as $x \to x_0$, we call $x_0$ a {\bf regular singular
  point}.\footnote{See the previous footnote. And irregular singular
  points are significantly more troublesome.}. The Method of Frobenius
lets us write down solutions near such points.

\newpage
\section{The main theorem}

(Thm 5.6.1 from Boyce \& DiPrima) Let  $x=0$ be a regular singular point of the differential
equation 
\begin{eqnarray}
  P(x)y'' + Q(x)y' + R(x)y &=& 0 \notag\\
  \label{fe}
  x^2y'' + x[x p(x)]y' + [x^2q(x)]y &=& 0
\end{eqnarray}

\noindent Then $xp(x)$ and $x^2q(x)$
are analytic at $x=0$ with convergent power series expansions

\begin{equation}
  xp(x)=\sum_{n=0}^{\infty}p_nx^n,\qquad  x^2q(x)=\sum_{n=0}^{\infty}q_nx^n
\end{equation}

for $|x|<\rho$ where $\rho>0$ is the minimum of the radii of
convergence of the power series for $xp(x)$ and $x^2q(x)$. Let $r_1$
and $r_2$ be the roots of the indicial equation

\begin{equation}
  \label{ind}
  F(r) = r(r-1) + p_0r + q_0 = 0
\end{equation}
 with $r_1 \ge r_2$ if $r_1$ and $r_2$ are real. Then in either the
 interval $-\rho < x < 0$ or the interval $0 < x < \rho$, there exists
 a solution of the form


\begin{equation}
  \label{y1}
  y_1(x) = |x|^{r_1}\left [ 1 + \sum_{n=1}^{\infty}a_n(r_1)x^n \right ],
\end{equation}

where the $a_n(r_1)$ are given by the recurrence relation:

\begin{equation}
  \label{recurrence}
  F(r+n)a_n + \sum_{k=0}^{n-1}a_k[(r+k)p_{n-k}+q_{n-k}]=0,\qquad n\ge1
\end{equation}

with $a_0=1$ and $r=r_1$.  If $r_1-r_2$ is neither zero nor a positive
integer, then in either the interval $-\rho < x < 0$ or the interval
$0 < x < \rho$, there exists a second solution of the form

\begin{equation}
  \label{y2bothreal}
  y_2(x) = |x|^{r_2}\left [ 1 + \sum_{n=1}^{\infty}a_n(r_2)x^n \right ].
\end{equation}

The $a_n(r_2)$ are also determined by the recurrence relation
\eqref{recurrence} with $a_0=1$ and $r=r_2$. The power series in
\eqref{y1} and \eqref{y2bothreal} converge at least for $|x|<\rho$.

If $r_1=r_2$, then the second solution is
\begin{eqnarray}
  \label{y2equal}
  y_2(x) &=& y_1(x)\ln|x|+|x|^{r_1}\left [ 1 + \sum_{n=1}^{\infty}b_n(r_1)x^n \right ] \notag\\
%  y_2(x) &=& y_1(x)\ln|x|+|x|^{r_1}\left [ 1 + \sum_{n=1}^{\infty}a^{'}_n(r_1)x^n \right ] \qquad a_n^{'} \equiv \d{a_n}{r}\bigg|_{r=r_1}. 
\end{eqnarray}

If $r_1-r_2 = N \in \N$, then
\begin{equation}
  \label{y2offbyN}
  y_2(x) = ay_1(x)\ln|x|+|x|^{r_2}\left [ 1 + \sum_{n=1}^{\infty}c_n(r_2)x^n \right ].
\end{equation}

The coefficients $a_n(r_1), b_n(r_1), c_n(r_2)$ and the constant $a$
can be determined by substituting the form of the series solutions for
$y$ into \eqref{fe}. The constant $a$ may turn out to be zero, in
which case there is no logarithmic term in the solution to
\eqref{y2offbyN}. Each of the series in \eqref{y2equal} and
\eqref{y2offbyN} converges at least for $|x|<\rho$ and defines a
function that is analytic in some neighborhood of $x=0$.  In all three
cases, the two solutions $y_1(x)$ and $y_2(x)$ form a fundamental set
of solutions for the given differential equation.

\section{Bessel Equations}

Bessel equations are perhaps the most widely used of the ``special''
functions (Bessel, Airy, Legendre, etc.) in mathematical physics. They
are most commonly seen in problems with circular or cylindrical
symmetry, including vibrations of a uniformly stretched circular
membrane (e.g. a drum head), quantum mechanical problems with circular
symmetry (e.g. this week's homework assignment), heat conduction in
cylindrical objects, diffusion problems on a lattice, etc.

They are the solutions to Bessel's equation:

\begin{equation}
  \label{bessel}
  x^2y'' + xy' + (x^2-v^2)y
\end{equation}

where $v$ is a constant, also called the \textbf{order} of the Bessel equation.

Given the form of \eqref{bessel}, you will not be surprised to find
that we plan on using the results of the preceding section in its
analysis. We will cover three cases here (order 0, 1/2 and 1). We will
see more applications in our triumphant return to series solutions
later in the class (i.e. solving PDEs by separation of variables).

\textbf{Problem 1} Is $x=0$ is a regular singular point of \eqref{bessel}?
\vskip2in

\textbf{Problem 2} What is the indicial equation? What are $r_1$, $r_2$?
\vskip2in
\subsection{Order Zero}
``order zero'', as mentioned above, means $v=0$:


\begin{equation}
  \label{b0}
  L[y] = x^2y'' + xy' + x^2y
\end{equation}


So, we will have $r_1=r_2=0$. As per the theorem, we know that we'll
have one solution of the form \eqref{y1}:

\begin{equation*}
  y(x) = \phi(x,r) = a_0x^{r} + x^r\sum_{n=1}^{\infty}a_nx^n.
\end{equation*}

\textbf{Problem 3}
After reminding yourself that
\begin{eqnarray}
  y(x) &=& \sum_{n=0}^{\infty}a_nx^{r+n} \notag\\
  y'(x) &=& \sum_{n=0}^{\infty}(r+n)a_nx^{r+n-1} \notag\\
  \label{derivs}
  y''(x) &=& \sum_{n=0}^{\infty}(r+n)(r+n-1)a_nx^{r+n-2},
\end{eqnarray}

\textbf{3a} plug \eqref{derivs} into \eqref{b0}. You should find that you'll need
to strip off the first two terms.
\vskip2in

\textbf{3b} what is the recurrence relation? Don't forget to simplify the denominator.
\vskip1in

You've seen these sorts of recurrence relations before. We'll ratchet
our way up to a general form, so we start with the first term. Recall
that $a_0$ is not zero.  Plugging in $r=0$, we see that the
coefficient of $x^r$ is zero, as it should be from the indicial
equation $F(r) = 0$. We are not surprised to find that we must set
$a_1$ to zero for the coefficient of $x^r+1$ to be zero. So, our
recurrence relation turns into

\begin{equation*}
  a_n(0) = -\frac{a_{n-2}}{n^2}
\end{equation*}

All of the odd terms are zero, and we write
\begin{equation*}
  a_{2m}(0) = -\frac{a_{2m-2}(0)}{(2m)^2}
\end{equation*}

\textbf{Problem 4} Write out $a_2(0), a_4(0), a_6(0)$.
\vskip2in

Looking at this, we see that we can split out $2^2, 2^4,2^6$ from the denominators, writing the general term as

\begin{equation*}
  a_{2m}(0) = \frac{(-1)^ma_0}{2^{2m}(m!)^2}, m=1,2,3,...
\end{equation*}

and so (in what we've previously decided is a unique and lucky situation), we can write out the first solution as

\begin{eqnarray}
  y_1(x) &=& a_0\left [ 1 + \sum_{m=1}^{\infty} \frac{(-1)^mx^{2m}}{2^{2m}(m!)^2}\right ] \notag\\
  \label{J0}
         &=& a_0\left [J_0(x) \right ]
\end{eqnarray}
where $J_0(x)$ is known as the \textbf{Bessel function of the first kind of order zero}. The results of Problem 1, combined with the theorem, tell us that $J_0$ is analytic at $x=0$ and that the series \eqref{J0} converges for all $x$.

Now we want to find $y_2(x)$. [Hint! This will give us $Y_0$, a Bessel function of the \textit{second} kind of order zero!]. In the general case, we will need to plug back in and solve for the $b_n$ terms. In this case, we are lucky enough that we can evaluate the derivative and just solve directly for $a_n{'}(0)$.

\end{document}
